%%%%%%%%%%%%%%%%%%%%%%acknow.tex%%%%%%%%%%%%%%%%%%%%%%%%%%%%%%%%%%%%%%%%%
% sample acknowledgement chapter
%
% Use this file as a template for your own input.
%
%%%%%%%%%%%%%%%%%%%%%%%% Springer %%%%%%%%%%%%%%%%%%%%%%%%%%

\extrachap{Management Summary}

ThyssenKrupp Industrial Solutions has engineers all around the world. Since their collaboration process is not aligned perfectly, the challenge is to find out how the collaboration of engineers can be improved. This challenge is tackled by a joint team of eight students in total from the University of St.Gallen and the Hasso Plattner Institute. To come up with a solution the methodology of Design Thinking is used. The following report is a summary of the last few weeks where we worked on the first two phases of the Design Thinking cycle: the Design Space and Critical Function Prototype phase.


During the Design Space phase (Chapter 2), we conducted a benchmark across several industries to see how other companies handle the issue of collaboration. Furthermore, we had a look into existing tools to get an overview about state of the art technologies. The benchmark already gave us first insights and needs of users with regard to team collaboration. Afterwards, we continued our study of insights and needs by conducting interviews within ThyssenKrupp. Recently, we were able to talk to engineers in Germany and India. All the knowledge we collected about the topic helped us to define two personas, which could be typical users at ThyssenKrupp.


By analyzing the needs and insights, we were able to identify critical functions and experiences (Chapter 3). They are the basis for the development of the prototypes. We selected ten different prototypes for this report and each of them fulfills one or several critical functions and experiences. Various users tested the prototypes afterwards, so we have been able to gather feedback and to learn more about them in order to be able to improve them or create new prototypes.


The project lasts until June 2015, so this report is just about the first findings. Our current activities will continue as we might need specific benchmarks for a certain topic for example, but the process of Design Thinking continues with the phases of Dark Horse and Funky Prototypes. Afterwards, our designs and prototypes will converge until a final prototype is presented in June 2015.

